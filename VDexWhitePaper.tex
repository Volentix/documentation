\documentclass[]{article}
\usepackage{url}
\usepackage{cite}
\usepackage{listings}
\usepackage{xcolor} % for setting colors
\usepackage{lmodern}
\usepackage{graphicx}
\usepackage{textcomp}
\usepackage{hyperref}
\usepackage{enumerate}
\usepackage[numbers]{natbib}

% set the default code style
\lstset{
	frame=tb, % draw a frame at the top and bottom of the code block
	tabsize=4, % tab space width
	showstringspaces=false, % don't mark spaces in strings
	numbers=left, % display line numbers on the left
	commentstyle=\color{green}, % comment color
	keywordstyle=\color{blue}, % keyword color
	stringstyle=\color{red} % string color
}

%opening
\title{VDex White Paper v1.0}
\author{
		The Volentix Labs Team\\
	\texttt{info@volentixlabs.com}
}

\begin{document}
\tableofcontents
\maketitle
\begin{description}
\item Copyright \textcopyright 2018 Volentix
\end{description}

{\tiny Without permission, anyone may use, reproduce or distribute any material in this white paper for non-commercial and educational use (i.e., other than for a fee or for commercial purposes) provided that the original source and the applicable copyright notice are cited.

DISCLAIMER: This VDex White Paper  is for information purposes only. The authors do not guarantee the accuracy of or the conclusions reached in this white paper, and this white paper is provided 'as is'. 
Volentix Labs do not make and expressly disclaims all representations and warranties, express, implied, statutory or otherwise, whatsoever, including, but not limited to: 
(i) warranties of merchantability, fitness for a particular purpose, suitability, usage, title or non infringement; (ii) that the contents of this white paper are free from error; and (iii) that such contents will not infringe third-party rights. Volentix Labs and its affiliates shall have no liability for damages of any kind arising out of the use, reference to, or reliance on this white paper or any of the content contained herein, even if advised of the possibility of such damages. In no event will Volentix or its affiliates be liable to any person or entity for any damages, losses, liabilities, costs or expenses of any kind, whether direct or indirect, consequential, compensatory, incidental, actual, exemplary, punitive or special for the use of, reference to, or reliance on this white paper or any of the content contained herein, including, without limitation, any loss of business, revenues, profits, data, use, goodwill or other intangible losses.}

\begin{abstract}

VDex is a decentralized exchange with the user and community in mind. 
Using some of the most recent paradigms and established protocols for security, 
ease of use and multi asset support, this low friction peer-to-peer exchange 
abides by open standards and ensures a harmonious and seamless flow among 
decentralized applications. Built with a collection of smart EOS.IO contracts, 
it contains easy to use options for security, anonymity, speed of payment, liquidity and profit margin.
Open order books support integration with other decentralized exchanges, 
in effect producing a massive decentralized exchange of exchanges,
 increasing the liquidity and effectiveness of all the connected exchanges.
VDex is also a pillar in the Volentix ecosystem, a network of DApps whose synergy results in even
higher liquidity for VDex users.
\end{abstract}
\section{Introduction}

VDex is a distributed exchange that will provide a highly customizable environment for speed, cost, 
anonymity, security, and scalability. 
Power users will feel enabled with the freedom to choose and thrive, 
while new users will feel welcomed and free from the risks inherent in a centralized system. 
The growth of the product will reside in a flexible architecture 
able to adopt the best practices of 2nd generation blockchain applications.  
The task of building VDex mainly resides in successfully merging
and integrating today's best protocols, paradigms and patterns to match the 
Volentix requirements on top of the EOS.IO decentralized operating system.
% Suggest removing
% To provide a flexible modular product, developers will need an environment that serves its purpose without burdening design with redundant complexity. 
%
 	
\section{Methodology}

All assumptions made in this paper will be verified by prototyping with our custom EZEOS software, built
with EOS.IO's \textbf{\textit{cleos}} command line tools. This software resides at: https://github.com/Volentix/ezeos

\section{Volentix}	
The Volentix ecosystem which consists of DApps which improve the effectiveness of each other.
The "four pillars" are an initial set of DApps which support the entire Volentix ecosystem,
each in a way specific to their own needs.
From the perspective of VDex, the purpose of Volentix DApps is to grow the user base of VDex
and thus increase the liquidity available to all users.

Venue: grows the Volentix community
Verto: enables funds to be continually maintained by the user while using VDex
Vespucci: increases trust in the tokens available on VDex
VDex: provides liquidity between all crytocurrencies

\subsection {Venue dynamic community platform}

Venue is a platform which brings together members of the Volentix community 
and to facilitate distribution of VTX and bring greater awareness to the Volentix Project. 

The first iteration of Venue is a platform for distributing VTX. 
VTX rewards can be earned in exchange for participating in community building campaigns, submitting bug fixes, or claiming bounties. 
Included in this webapp are leaderboards and live metrics showing user participation. 
The first signature campaign was launched on the \textit{https://bitcointalk.org/} forum on July 13th, 2018. 
Visit \textbf{\textit{https://venue.volentix.io}} for more information. 
	
\begin{figure}
	\centering
	\includegraphics[width=0.7\linewidth]{../Downloads/white_background-ecosystem02}
	\caption{}
	\label{fig:whitebackground-ecosystem02}
\end{figure}



\subsection {Verto wallet}
Verto is multi-currency wallet for VDex. 
It is the central component for all Volentix platforms and is necessary to ensure users continually and safely maintain custody of their private keys.
Verto employs a system of smart contracts to maintain the state between two trading clients, the simplest operations being done with atomic swaps. 

\subsection {Vespucci analytical engine}
Vespucci is an application that provides information and analytics about tradeable digital assets, 
as well as a tool to graph and compare tradeable digital assets. 
Vespucci increases liquidity by boosting user confidence and bringing more users to the Volentix platform.
Vespucci aggregates data presently scattered amongst blockchain explorer sites, chat rooms, and websites, 
as well as using historic trading records, forum sentiment and plan analysis, trading trends, developer activity,
digital asset distribution and governance information, terms and conditions, and historic address balances.




\subsection {VDex}

VDex will provide crypto currency exchange services directly from the user's Verto wallet where both public and private keys are locally managed.
The swapping of crypto assets does not involve the temporary custody of the tokens by a central operator, 
therefore VDex also allows users to trade crypto assets with others without losing custody of the tokens they hold. 
Furthermore, VDex allows digital tokens to remain on the originator's wallet until all required transactions are completed. 
VDex supports trading in many cryptocurrencies and hosts multiple liquidity pools to accomodate different types of markets.


									
\section{Architecture}
	
	\subsubsection{Overview}
	Cross-chain, inter-wallet transactions are done by providing wallets with the contracts or scripts to transact in any cryptocurrency. 
	The transaction between Bob and Alice involves each sending funds toward the other's accounts 
	using these provided contracts best described in the original atomic swap paper.\cite{22} 
	These contracts have guarantees toward each other by the means of shared secrets within timeouts or refund occurs. 
	The various means of ensuring collateral for a particular transaction or augmenting liquidity of a network 
	will be described later in this document.
  
	\subsubsection{Operating system}
	EOS.IO is an operating system-like framework upon which decentralized applications can be built. 
	The software provides accounts, authentication, databases, asynchronous communication, and scheduling across clusters. 
	Components and protocols are already built into the platform, and just a subset can be used to satisfy VDex requirements. 
	VDex will initially benefit from the standard features offered by EOS.IO such as account and wallet creation 
	and the recovery of stolen keys, but will subsequently implement the protocols 
	for the creation of decentralized exchanges through its contracts and provided tools	\cite{3}\\
  
	\begin{enumerate}
					
			\item \textbf{Context Free Actions} \\
		Most of the scalability techniques proposed by Ethereum (Sharding, Raiden, Plasma, State Channels) 
		become much more efficient, parallelizable, and practical while also ensuring speedy inter-blockchain communication and unlimited scalability.
		A Context Free Action involves computations that depend only on transaction data, but not upon the blockchain state.
		ex: Parallel processed signature verifications\\
		
		\item\textbf{ Binary/JSON conversion} \\
		 EOS contracts combine the human readability of JSON with the efficiency of binary. \\
	
		\item \textbf{Parallelisation and optimisation\\ } 
		Separating authentification from application allows faster transaction times and increases bandwidth.
		EOS.IO blocks are produced every 500 ms.
		
	\item \textbf{Web Assembly(WASM)}   \\
	Web assembly enables high-performing web applications while also secures each applicatios in its own sandboxe.
	This will provide VDex with network access and
	filesystem namespace restrictions, enforced rule-based execution, and better control over what processes are spawned.
	
	\item \textbf{Rust/C++ contracts\\}
	At the time of this writing,
	C++ has best tooling for and execution speeds for WASM.
	C++ is a much more mature language than for instance Solidity used for Ethereum contracts.
	It also has better debugging support as well as libraries that have been tested over the years and provide reliable functionality. 
	The EOS codebase has also very heavy usage of templates.
	For example, C++ allows the use of templates and operator overloading to define a runtime cost-free validation of units.
	Managing memory is much easier building smart contracts because the
	program reinitializes to clean state at the start of every message. Furthermore, there is rarely a need to implement dynamic memory allocation. 
	The WebAssembly framework will automatically reject any transaction addressing memory wrong.
	Since EOS.IO contracts use C++14 one can resort to smart pointers if dynamic memory allocation is needed.
	The first implementation of PARSEC will be in Rust.\cite{23}
	 	
	\item \textbf{Schema defined messages and database} \\
	Service contracts are standardized to guarantee a baseline measure of interoperability associated with the harmonization of data models.
	The \textit{Standardized Service Contract} design principle advocates that service contracts be based on standardized data models. 
	Analysis is done on the service inventory blueprint to find out the commonly occurring business documents that are exchanged between services. 
	These business documents are then modeled in a standardized manner. 
	The Canonical Schema pattern reduces the need for application of the data model transformation design pattern.
	\cite{1}
	
		
\end{enumerate}
	 
	\subsubsection{Inter Contract Communication}
	Data is shared between contracts via an oracle which creates a transaction embedding the data in the chain. 
	"An oracle, in the context of blockchains and smart contracts, is an agent that finds and verifies real-world occurrences 
	and submits this information to a blockchain to be used by smart contracts."\ 
	\cite{2}
	Every node will have an identical copy of this data, so it can be safely used in a smart contract computation.
	Oracles push the data onto the blockchain rather than the smart contract pulling the information.
	Most reading of the data is done via polling \textbf{nodeos} (the blockchain instance) to monitors the blockchain's state 
	and performs certain actions in response. 
	
	
	\subsubsection{Side Chains}
	In EOS.IO, issuing tokens is a process which creates a sidechain. 
	Sidechains are emerging mechanisms that allow tokens and other digital assets from one blockchain 
	to be securely used in a separate blockchain and then be moved back to the original blockchain if needed. 
	There can be multiple side chains where different tasks are distributed accordingly for improving the efficiency of processing. 	
	For inter-blockchain communication, the EOS.IO protocol creates a TCP like communication channel between chains to evaluate proofs.
	For each shard (a unit of parallelizable execution in a cycle), a balanced merkle tree is constructed of these action commitments 
	to generate a temporary shared merkle root; 
	this is done for speed of parallel computation. 
	The block header contains the root of a balanced merkle tree whose leaves are the roots of these individual shard merkle trees.
	\cite{3} 

	
	\subsubsection{Liquidity}
	In order for a token to effectively partake in the global token
	economy, its trading volume must cross a critical barrier where
	the matches between buyers and sellers become frequent enough to ensure a stable pool of "coincidences of wants" \cite{10}. 
	This reliability within an exchange is known as liquidity. We say a token is liquid if it is easily possible to buy or sell it without considerably affecting its price.
%DL	from eosio.boot telegram chat
%	We introduce a new token pegging algorithm that provides high liquidity and narrow spreads, while being robust against default in the event collateral loses value. We utilize the Bancor algorithm to provide liquidity for market participants. Market trading fees generate profits for the dollar-short positions and automatically replenish the collateral ratio if the collateral falls. With sufficient trading volume, high initial collateral ratio, the algorithm can work with highly volatile collateral assets.  In the event the value of all collateral falls below the total outstanding derivatives the Bancor algorithm will continue to make the market earning fees until the price recovers. The premise behind the algorithm is that there market trading fees for a highly-in-demand asset can more than offset any capital losses associated with maintaining a market peg. Furthermore, it should be profitable for those willing to be slightly leveraged-long in the collateral asset to contribute to the liquidity fund and increase liquidity. Our algorithm uses the median price feed of a number of oracles to keep the peg within a few percent of the pegged asset value.
	
	\begin{enumerate} 
		\item Implementation of the Loopring protocol with the use of EOS.IO contracts acting as nodes.\cite{7}
		\item Implementation of the Bancor algorithm used to bring stability to the token.\cite{10}
%		EOS.IO uses the Bancor algorithm  for managing ressources.
%		The goal is to introduce a new token pegging algorithm that provides high liquidity and narrow spreads, 
%		while being robust against default in the event collateral loses value. 
%		Bancor algorithm provides liquidity for market participants. 
%		Market trading fees generate profits for the dollar-short positions and automatically replenish the collateral ratio if the collateral falls. 
%		With sufficient trading volume, high and initial collateral ratio, the algorithm can work with highly volatile collateral assets.\cite{25}
		\item Toggles between these protocols and atomic swaps(HTLC) according to the Vespucci analysis on the VDex network.
		\end{enumerate}
	
	\subsubsection{Hashed timelock contracts (Atomic Swaps)}
	A Hashed Timelock Contract (HTLC)\cite{22} is a type of smart contract enabling the implementation of time-bound transactions.
	Users will be offered a variable lock-in period for their transations, 
	with a discount on the transaction fee in exchange for choosing a slightly greater lock-in period.
	
	\subsection{Network Topology}
	\subsubsection{Nodes}
	Nodes are the endpoints of the VDex network.
	Their function is to:\
	\begin{enumerate}
		\item Act as a portal to VDex trough the Verto wallet.
		\item Merge order book information with others.
		\item Settle order book.\
		\item Manage order cancellation.\
		\item Assign timeouts for the \textbf{Raft} Protocol.\
		\item Initiate contract for orders that have been filled.
	\end{enumerate}

Nodes earn a portion of the fee for each transaction.
If a user has sufficient funds and posessing a good track record, their Verto wallet can act as a node.
	
	\subsubsection{Aggregators}
	
	The VDex aggregators are dedicated Volentix servers for simulator and security purposes. 
	One of their functions is to pull logs and order book data from nodes into sparse distributed representations for hierarchical temporal memory
	as intrusion \cite{24} analysis for detecting anomalies in the system such. 
	The aggregators will also be host to other components such as metachai  ledgers\cite{20}and blockchain scrapers.
	 
	
	\subsubsection{Latency}	
	EOS.IO has low latency block confirmation (0.5 seconds).\cite{3}
	This latency can be provided if the currencies being traded are issued from blockchains that are equally fast, 
	otherwise the transaction is as fast as the slowest block chain
	(for example a bitcoin block takes 9 minutes to mine at time of this writing). 
	On completion of the transaction, a transaction receipt is generated. 
	Receiving a transaction hash does not mean that the transaction has been confirmed; it means only that a node accepted it without error, 
	although there is also a high probability other producers will accept it. 

\subsection{OrderBook}
An order book is the list of orders that VDex will use to record the interest of buyers and sellers. 
A matching engine uses the book to determine which orders can be fulfilled.
The Loopring protocol allows for customizing the order book data structure according to data requirements. \cite{7} 
With this in mind, containers provided by EOS.IO can be used for optimal performance.\cite{25}

\subsubsection{Data structures}
\begin{enumerate}
\item Loopring FIFO\
	First-in first-out circular buffer, as suggested by the protocol. 
	Nodes can design their order books in any number of ways to display and match a user's order. 
	Follows an OTC model, where limit orders are positioned based on price alone.
	\cite{7}  
\item EOS.IO persitence API
	The order book will take advantage of the powerful multi-index container shared among nodes through the same EOS.IO account.
\end{enumerate}
%\subsubsection{Operations}
%Iterating over the orderbook container:
%\begin{enumerate}
%\item Remove the existing orders if no match is found.(Order Matching)
%\item Remove a cancelled order.
%\item Add a new order that didn’t fully match.
%\item Merge with neighboring books or selected book.
%\item Initiate contract for orders that have been filled.
%\end{enumerate}

\subsubsection{On-Chain order book}
An on-chain order book is a record of offers residing on the wallet (node) chosen to settle
the order book. It resides in a persistent database on each node subscribing 
to the same account as all the other nodes.

\subsubsection{Off-Chain order book}
Residing on the aggregator, offline order books will serve for simulator and security purposes.


\subsubsection{Decentalization process of order book settlement}
To ensure decentralization, nodes will take turns to settle the Order Book. 
The settling node must be designated by the protocol and all order book entries from all nodes must be available to the settling nodes. 
The RAFT\cite{18} and PARSEC\cite{23} protocols offer elegant and simple solutions. 
The concepts of RAFT are easy to implement and have been around since the time of PAXOS\cite{24}, 
while PARSEC is fairly recent but more efficient, using directed acyclic graphs, eliminating the need for copying logs.

\subsection{Order settlement}
In the case of a FIFO container for the order book,
settling occurs as such:
\begin{enumerate}
	\item Check for subrings
	this operation promotes security by checking the order of the ring is the same as the initial one.
	\item Load Balancing (Fill rate/stock availability)
	\item Rate is equal or less than the original buy rate set by user (Definition of limit orders)
	\item Process cancellations\\
	\item Orders are scaled according to the history of filled and cancelled amounts\\	 
\end{enumerate}	
	

\subsection{VTX}
VTX is the cryptocurrency used on the vDex exchange. 
It can be used within all of the four pillars of the Volentix system:
\begin{enumerate}
\item To pay transaction fees on the VDex.
\item To invest. 
\item To vote on proposals submitted to the network, using the voting rights allocated to VTX holders.
\item To submit proposals to the network.
\item To stake support for project building, liquidity, or proposal reviews. 
\item As a fee redistribution token on the VDex exchange.
\item To incite users to partake in order book settlement.
\item To reward participants in the consensus process and in Venue campaings.
\end{enumerate}
%EOS user has VTX 
%ETH	user has ETH
%Burn VTX on EOS and create VTX on etherium									 				
%sell ETH for VTX take ETH into VTX zone on etherium
%sell VTX for ETH
%Bitcoin script
%Bitcoin counterparty
%HTLC
%NEO
%Qton
%Bitcoin cash



\subsubsection{Token creation}
The \textbf{eosio.token} contract from the EOS.IO framework will be used to issue 2.1 billion EOS.IO compliant tokens 
with a supply of 1.3 billion.
 
 
\subsubsection{Initial Token offering}
 
Distribution
\begin{enumerate}
\item \textbf{The founders}\\
The initial promoters of the project. 
Conceptualization, plan and initialization.	
\item \textbf{Prior work,	 core team }\\
Consists software engineers, blockchain analysts, legal experts and business development specialists. 
In addition to the core team, Volentix has a board of advisors consisting of top experts from a variety of fields relevant to the project.
\item\textbf{ Decentralized treasury}\\
Publicly controlled funds for community projects, business operations and salaries for ongoing development. 
A small percentage or every transaction goes back into the treasury.
\item \textbf{Ongoing core development} \\
Incentives for core team. These are bonuses given throughout the project after milestones have been reached.
\end{enumerate}
	
\begin{table}[h!]
	\begin{center}
		\caption{Token distribution}
		\label{tab:table1}
		\begin{tabular}{l|c|r}
			\textbf{Percentage of issued tokens} & \textbf{price} & \textbf{Timelock} \\
		
			\hline
			35\% Future Decentralized Treasury & & \\
			\hline
			5\%  Initial funding & 0.000016-0.000020 BTC & \\
			\hline
			28\% Distribution & 0.000021 BTC+ & \\
			\hline
			12\% Founders &  & yes \\
			\hline
			10\% Prior Work, Team and Advisors & & yes\\
		
			\hline
			10\% Ongoing Core Development enticement & &\\
			\hline		
			
				
		\end{tabular}
	\end{center}


\end{table}

\subparagraph{Justification of token crowdsale vote}
Pre-sale VTX holders, founders and the core team will be given the opportunity to participate in an \textbf{\textit{augmented} \textit{vote}}\cite{21} 
to know if the Volentix community wants a crowdsale or not. 
Augmented votes will require voters to take an easy test which proves they acknowledge facts concerning crowdsales. 
Participants to the elaboration of this test grid will compile facts linked to publications regarding crowdsales and will earn VTX in the process. 
The resulting vote will therefore be \textit{\textbf{augmented}} by ensuring all voters have the same established knowledge regarding this issue.  

\subparagraph{Multi signature account}
The raised funds will be deposited to a multi signature account owned by the founders 
that can only be accessed with the consent of at least two of three signees.

\subsubsection{Fee model}
\subparagraph{Transaction fees\\}
\begin{enumerate}
	\item Fees will be collected in any currency.\\
	\item Fees can be assigned to specific accounts\\
	\item A small fee collected for general maintenance by developers\\
\end{enumerate}
\subparagraph{Lock-in period fee\\}
	A Hashed Timelock Contract (HTLC)\cite{22} is a type of smart contract enabling the implementation of time-bound transactions.
\begin{enumerate}
	\item Users can lock in funds for 24hrs and have a free transaction\\
	\item Users can lock in funds for 24hrs+ and generate VTX \\
\end{enumerate}
\subparagraph{Order book settlement fees}
A wallet can become a node and earn VTX through order book settlement. 


\subparagraph{EOS.IO}
The following considerations will be applicable for deploying the exchange on the EOS.IO platform
\begin{enumerate}
\item Deploying a contract has a cost but is free to use. 
\item Developers have to stake EOS.IO tokens to deploy contract.
After the contract is destroyed, the locked tokens are returned.
\item DApps must allocate resources to their contracts, memory, cpu, bandwidth. 
\item The choice of who pays the resources is up to the DApp.
\item Multiple messages in one transaction and multiple accounts can be asigned to the same thread.

\end{enumerate}

\subparagraph{Budget forecast}
Since the acceptance of decentralized exchanges has not yet arrived, but is deemed to happen, 
it is logical to say that the value of VTX will increase with demand over a longer period of time, around 5 years.

\subsection{Inter blockchain communication}
EOS.IO is designed to make Inter-Blockchain-Communication proofs lightweight. 
For chains with an insufficient capacity for processing the  IBC proofs and establishing validity, 
there is the option to degrade to trusted oracles/escrows.
To directly control other currency transactions with an
EOS.IO based smart contract, a trusted mutisig wallet holding the currency 
in escrow is used to persuade the signing/publishing of the currency 
transaction based on IBC proofs from the originating chain.	
\subsection{Security model}
\subsubsection{Introduction}
The assumptions made in this section are made from the analysis on gathered information. 
Full security testing and securing will begin after the ongoing prototyping phase. 
The security concerns with the VDex environment can generally be categorized as variants of the following scenarios:
	\begin{enumerate}
		\item The attacker executes malicious code within a transaction
		\item The order of transactions is manipulated
		\item The timestamp of a block is manipulated
	\end{enumerate}
	\subsubsection{General actions to be taken}
		\begin{enumerate}	
		\item{Filter incoming data} 
		\begin{enumerate}
			\item Data from blockchains and order book resolution.
			\item Data received by nodes
			\item Data received by wallet
			\item Data received by aggregators
		\end{enumerate}
		\item {Filter outgoing data}
		\begin{enumerate}
			\item Cancel order
			\item Commit
		\end{enumerate}
	\end{enumerate}
		\subsubsection{Contract security}
		\begin{enumerate}
		\item Keep vast majority of funds in a time-delayed, multi-sig controlled account
		\item Use multi-sig on the hot wallet with several independent processes/servers double-checking all withdrawals
		\item Deploy a custom contract that only allows withdrawals to KYC'd accounts and require multi-sig to white-list accounts
		\item Deploy a custom contract that only accepts deposits of known tokens from KYC'd accounts
		\item Deploy a custom contract that enforces a mandatory 24-hour waiting period for all withdrawals
		\item Utilize contracts with hardware wallets for all signing, even automated withdrawals
		\item Seamless sytem to upgrade broken contracts
		\item Ability to pause the functionality of a contract 
		\item Ability to delay an action of a contract	
		\end{enumerate}
		\subsubsection{Malware detection by auditing processes}
		The system will provide insights on rogue processes during the transaction period with AI analysis residing on the aggregators. 
		\subsubsection{Random diversification}
		By using the RAFT protocol in the election process, a certain level of randomization is aquired with varying length of timeouts. 
		The toggling of protocols is a level of complexity 
		\subsubsection{Multiple factor identification}
		As in many existing applications, this measure is efficient and already known to the public at large.
		\subsubsection {Logs}
			Ensure inspection of logs as control.
			\begin{enumerate}
				\item Raft.
				\item Anomaly detection with AI(Numenta).
				\item Script investigations of certain non token purchases related addresses.
			\end{enumerate}
		\subsubsection{Transaction as Proof of Stake (TaPoS)}
			\begin{enumerate}
				\item Prevents a replay of a transaction on forks that do not include the referenced block 
				\item Signals the network that a particular user and their stake are on a specific fork.
			\end{enumerate}
		\subsubsection{Double spend}
		A double spend is an attack where a given set of coins is spent in more than one transaction.
		\begin{enumerate}
			\item Send two conflicting transactions in rapid succession into the network. This is called a race attack. 
			\item Pre-mine one transaction into a block and spend the same tokens before releasing the block to invalidate that transaction. 
			This is called a \textit{Finney} attack.
			\item Own 51+\% of the total computing power of the network to reverse any transaction, 
			as well as have total control of which transactions appear in blocks. 
			This is called a 51\% attack.
			This is impossible according to EOS.IO, Loopring or Raft.
			If a block producer takes an unreasonable amount of runtime or is not profitable enough, the process is blacklisted.\cite{7}
		\end{enumerate} 

		\subsubsection{Front running}
		To prevent someone from copying another node's trade solution 
		and have it mined before the next supposed transaction in the pool, 
		a higher fee per transaction is charged.\\ 
		The major scheme of front-running in any protocol for
		order-matching is order-filch: when a front-runner steals
		one or more orders from a pending order book settlement
		transaction. EOS.IO and loopring both have remedies to this.
		In both cases keys are not part of the on-chain transaction and thus remain unknown to parties other than the order book settling node.
		Nodes dictate how they manage orders;
		each node will use a different solution for resolving the order books, inducing another level of randomness to promote security.
			
		
		\subsubsection{Forged identities}
		Malicious users acting as themselves or forged identities 
		could send a large number of small orders to attack Loopring
		nodes. However, most of these orders will be rejected 
		for not yielding satisfying profit when matched. 
		Again, nodes should dictate how they manage orders.
		
		\subsubsection{Insufficient Balance}
		Malicious users could sign and spread orders whose 
		value is non-zero but whose address actually has zero 
		balance. Nodes could monitor and notice that some orders' 
		actual balance is zero, update these order states accordingly
		and then discard them. Nodes must spend time to update
		the status of an order, but can also choose to minimize the
		effort by, for example, blacklisting addresses and dropping
		related orders.
		
		\subsubsection{Timing attack}
		 Timing attacks are a class of cryptographic attacks through which a third-party observer can deduce the content of encrypted data 
		 by recording and analyzing the time taken to execute cryptographic algorithms.
		 The randomness of timeouts in the raft algorithm prevents this.
		
		\subsubsection{Other EOS.IO security attributes}
		\begin{enumerate}
		\item No uses of mutex or locks for on-chain parallellisation
		\item All accounts must only read and write in their own private database
		\end{enumerate}
		
	\subsection{Inter blockchain communication}
	
	Transactions sent to a foreign chain will require some facilities on the foreign chain to be trustless. 
	In the case of two EOS.IO based chains, the foreign blockchain will run a smart contract which accepts block headers and incoming transactions 
	from untrusted sources and is able to establish trust in the incoming transactions if they are provably from the originating chain. 
	For chains with an insufficient capacity for processing the IBC proofs and establishing validity, the options degrade to trusted oracles/escrows.
	
	
	\subsection{Multi blockchain}
	Multi-blockchain information can be obtained by aggregating blockchain timelines in parallel order 
	(with variance in the frequency of when the state is changed) into a comprehensible data structure.
	This will enable a system to trigger multichain load balancers, transfer states, draw data outputs from smart contracts, 
	and trigger execution of transactions on foreign blockchains. 
	Relative block distance, relative global state, and timestamped events are recorded on a global ledger to optimize and confirm transactions 
	before they actually happen on the native chain.
	This could be used to determine block production coincidence between chains to choose optimal liquidity.\cite{20}
	
	\subsection{User experience}
	\subparagraph{Simulator}
	In an effort to provide a better and safer user experience, 
	a VDex trading simulator will be provided with convenient scenarios.
	\subparagraph{Templates}
	Easy to use templates for standard transactions will also be provided.
	\subparagraph{Comprehensive customizeable and detailed interface}
	\begin{enumerate}
		\item Shows the entire market and fluctuations
		\item Shows wallet: balance and previous transactions.
		\item Shows detailed history with built in tax calculator.
		\item Contains toggles for advanced features.		 
	\end{enumerate}
	
	\subsection{True decentralization}
	 Decentralization is the process by which planning and decision-making are distributed or delegated away 
	 from a central, authoritative location or group.
	 
	 EOS.IO is a free, open source, scalable infrastructure for decentralized applications. 
	 It tries to ensure a fair and transparent block producer (BP) election process 
	 utilizing a democratic delegated proof of stake (DPoS) consensus.
	 A pragmatic understanding of the complexity of the task at hand combined with the unpredictability of human interactions 
	 implies that a small amount of centralization may be required in the initial stages.
	 The criteria of decreasing the \textit{incidences of centralization} with time is necessary. 
	 VDex will always vote and have direct influence in the direction of pragmatic decentralization.
	 
	 Regardless, the exchange itself will possess several mechanisms to promote decentralization. 
	 For instance the system for electing nodes when solving order books will not use a shared central clock or (DPoS) 
	 but will be based either on random timeouts for the determination of leaders in an election (RAFT), 
	 or using graph theory (DAG) in the case of the PARSEC protocol.
	 
	\subsection{System recovery}
	By ensuring that the latest version of the ledger is always put forward, 
	the RAFT and PARSEC protocols provide a robust system for recovery in the case of node failure.
	Security measures are also provided in case of IBC for trading with native blockchains. 
	In the case of non-identification of the chain, a fall back occurs to the native rules to wait until the next block or default to a short time lock.
	
	\subsection{Scalable and modular architecture}
	To secure the potential for innovation, the principles, 
	concepts and paradigms proposed by components of the system
	must favour decoupling of technologies. 
	Since creating and maintaining distibuted and decentralized systems
	is very complex we must use different strategies: 
	\subsubsection{Problem decomposition}
	Problem solving strategy of breaking a problem up into a set of subproblems, solving each of the subproblems, 
	and then composing a solution to the original problem from the subproblem solutions.
	\subsubsection{Minimize state space}
	Dynamic programming and templating are hard because of complexity and debugging challenges. 
	Nesting conditions can also seem unimportant for normal program execution.
	Special attention to these patterns and details in the initial design will maximize efficiency while allowing for easy replacement or addition of components.   
	In the context where CPU, bandwidth and RAM are monetized, this is even more important. 
	
\section{Risk}
	The sheer number of transactions VDex will eventually process is hard to visualize, 
	but in the context of the growing interest for decentralized exchanges, more transactions equate with more risk. 
	Managing the risks of handling currency, initially and still today, proves to be a challenge to centralized providers. It is long and arduous process to aquire
	the trust of a network.\\
	 The evolution of decentralized exchanges should not be expected to be any different, but unlike centralized exchanges, 
	dexes can have the support of the open source developer community which quickly contributes forward solutions to resolving problems and enhancing the product. \\
	
	% Suggest removing
	%\paragraph{}  The Volentix DAO will ensure VTX counsel nomination within assembly. 
	%This structure poses a risk by its novel and sparsely tested approach but ensures solidity by analysis of its theoretical precepts.
	%
\paragraph{}
\paragraph{}	

		
\section{Conclusion}
Constructs, concepts and protocols that stand out by their simplicity
while retaining their effectiveness will help implement a modular
design which will enhance the system with the capacity to easily add or replace components 
with the prospects of carefully advancing functionality as micro services are created.
Although certain assumptions made in this paper still remain to be verified,
a very distinctive direction for VDex architecture will be distilled as a highly flexible and modular MVP 
capable of adaptation and reaction to the changing technological ecosystem.
For this, provisions of the EOS.IO operating system, Loopring, Bancor, RAFT and PARSEC protocols have been retained. 

\section*{Bibliography}
\bibliography{bibl}
\bibliographystyle{siam}

\cite{1}
\cite{2}
\cite{3}
\cite{4}
\cite{5}
\cite{6}
\cite{7}
\cite{8}
\cite{9}
\cite{10}
\cite{11}
\cite{12}
\cite{13}
\cite{14}
\cite{15}
\cite{16}
\cite{17}
\cite{18}
\cite{19}
\cite{20}
\cite{21}

%Dash
%Omni
%NXT
%forging formula
%VertCoin
%Multiple algos for mining
%Verge (multiple algos) I2P Tor
%Bastille
%Ark
%Halo
%unicoin
%Unocoin
%Polka Dot
%Hashgraphs

 
\end{document}
